\latline
  {[\textbf{0}] Publius Scipio, ex virtute Africanus dictus, Iovis filius credebatur.}
  { Publius Scipio, on account of his virtue called Africanus, it was believed was a son of Jupiter. }
  {150}
  { CorrectedTranslation }
  { Notes }


\latline
  {Nam antequam conciperetur, serpens in lecto matris eius apparuit, et}
  { For before he conceieved, a snake appeared in the bier of his mother, and }
  {150}
  { CorrectedTranslation }
  { Notes }


\latline
  {postea ipsi parvulo draco circumfusus nihil nocuit.  In Capitolium }
  { after that a snake was found encircling the child and did no harm.  On the Capitoline }
  {150}
  { CorrectedTranslation }
  { Notes }


\latline
  {intepesta nocte euntem numquam canes latraverunt.  Nec hic}
  { in the depth of night, the dogs never barked an the arriving man.  No one before this man }
  {150}
  { CorrectedTranslation }
  { Notes }


\latline
  {quicquam prius coepit quam in cella Iovis diutissime sedisset, quasi}
  { began was there one who would be seated in the nave of Jove for a long time, as if }
  {150}
  { CorrectedTranslation }
  { Notes }


\latline
  {[\textbf{5}] divinam mentem acciperet.  Decem et octo annos natus, patrem apud}
  { he was taking in the divine wishes.  In his eighteenth year, at the battle of  }
  {150}
  { CorrectedTranslation }
  { Notes }


\latline
  {Ticinum singulari virtute servavit.  Postea, clade Cannensi nobilissimos}
  { Ticinium he saved his father's life with his exceptional virtue.  Afterward, in the time of the fiasco of Cannensis }
  {150}
  { CorrectedTranslation }
  { Notes }


\latline
  {iuvenes, Italiam desere cupientes, sua auctoritate compescuit et}
  { he compelled, by means of his own personal power, the most noble youth, the youth desiring to desert Italy }
  {150}
  { CorrectedTranslation }
  { \begin{enumerate}
  	\item compesco, -ere, ui:  to supress, to control
  \end{enumerate} }


\latline
  {incolumes per media hostium castra Canusium perduxit.}
  { and lead them unharmed through the middle of the the Canusian battle-camps of the enemy. }
  {150}
  { CorrectedTranslation }
  { Notes }


\latline
  {Viginti quattor annos natus, cum imperio proconsulari in Hispaniam}
  { In his twenty-fourth year, he was sent with proconsular imperium against Hispania and }
  {150}
  { CorrectedTranslation }
  { Notes }


\latline
  {[\textbf{10}] missus, Novam Carthaginem, qua die venit, cepit.  Ibi virginem}
  { he took New Carthage in the same day that he came.  There  beautiful virgins }
  {150}
  { CorrectedTranslation }
  { Notes }


\latline
  {pulcherrimam, ad cuius aspectum concurrebatur, ad se vetuit}
  { were running to his presence, but he foreswore himself to be joined to them }
  {150}
  { CorrectedTranslation }
  { Notes }


\latline
  {adduci patrique eius reddidit.  Cum Hasdrubalem Magnonemque,}
  { and returned to his own country.  Since Hasdrubal and Magno,}
  {150}
  { CorrectedTranslation }
  { Notes }


\latline
  {fratres Hannibalis, ex Hispania expulisset, victor ex Hispania domum}
  { brothers of Hannibel, had been expelled from Hispania, the victor returned home from  }
  {150}
  { CorrectedTranslation }
  { Notes }


\latline
  {regressus et consul ante annum suum factus, in Africam classem}
  { Hispania and having been made consul before his own (minimal) year (of maturity), lead across  }
  {150}
  { CorrectedTranslation }
  { Notes }


\latline
  {[\textbf{15}] traiecit.  Ibi amicitiam cum rege Numidarum Massinissa iam coniunxerat.}
  { an army against Africa.  There he had already joined himself with the king of the Numidians, Massinissa. }
  {150}
  { CorrectedTranslation }
  { Notes }


\latline
  {Cuius auxilio usus, Hannibalem ex Italia revocatum superavit}
  { Benefitted by his help, he bested Hannibal, recalled from Italy }
  {150}
  { CorrectedTranslation }
  { Notes }


\latline
  {et victis Carthaginiensibus leges imposuit.}
  { and on the conquered Carthaginians laid laws. }
  {150}
  { CorrectedTranslation }
  { Notes }


\latline
  {Decem post annos, bello Antiochi, legatus fratri fuit atque filium}
  { After ten years, during the Antiochan war, he was a legate of brother  }
  {150}
  { CorrectedTranslation }
  { Notes }


\latline
  {suum a rege captum gratis recepit.  Romam reversus, fratre a Catone}
  { and took back his own captured son as a favor from the king.  Having returned to Rome, with his brother }
  {150}
  { CorrectedTranslation }
  { Notes }


\latline
  {[\textbf{20}] repetundarum accusato, librum rationum in conspectu populi scidit.}
  { he was accused of extortion and tore apart the book of figures in the sight of the populus. }
  {150}
  { CorrectedTranslation }
  { Notes }


\latline
  {``Hac die,'' inquit, ``Carthaginem vici.  Quare in Capitolium eamus et}
  { ``On that day,'' he began, ``I conquered Carthage.  For that reason let us go to the Capitoline'' }
  {150}
  { CorrectedTranslation }
  { Notes }


\latline
  {diis suppliciemus.''  Inde, populum ingratum indignatus, in voluntarium}
  { and give praise to the gods.''  Thereafter the ungrateful population was outraged and }
  {150}
  { CorrectedTranslation }
  { Notes }


\latline
  {exsilium concessit, ubi reliquam egit aetatem.  Moriens autem}
  { he went into voluntary exile where he lead forgotten ages.  However, dying, }
  {150}
  { CorrectedTranslation }
  { Notes }


\latline
  {petiit ab uxore ne corupus suum Romam referretur.}
  { he sought from his wife that his body not be borne back to Rome. }
  {150}
  { CorrectedTranslation }
  { Notes }


