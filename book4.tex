\chapter*{Chapter 4} % (fold)
\label{sec:chapter_4}

\latline
  {Proca, rex Albanorum, duos filios habuit, Amulium et Numitorem }
  { Translation }
  {150}
  { CorrectedTranslation }
  { Notes }


\latline
  {quibus regnum annuis vicibus habendum reliquit. Sed Amulius fratri}
  { Translation }
  {150}
  { CorrectedTranslation }
  { Notes }


\latline
  {imperium non dedit et ut eum subole privaret, filiam eius, Rheam }
  { Translation }
  {150}
  { CorrectedTranslation }
  { Notes }


\latline
  {Silviam, Vestae sacerdotem praefecit, ut virginitate perpetua teneretur.}
  { Translation }
  {150}
  { CorrectedTranslation }
  { Notes }


\latline
  {[\textbf{5}] Quae a Marte compressa Remum et Romulum edidit.  Amulius }
  { Translation }
  {150}
  { CorrectedTranslation }
  { Notes }


\latline
  {ipsam in vincula compegit, parvulos in Tiberim abiecit, quos aqua }
  { Translation }
  {150}
  { CorrectedTranslation }
  { Notes }


\latline
  {in sicco reliquit.  Ad vagitum lupa accurrit eosque uberibus suis aluit. }
  { Translation }
  {150}
  { CorrectedTranslation }
  { Notes }


\latline
  {Mox Fastulus pastor eos collectos Accae Laurentiae coniugi educandos }
  { Translation }
  {150}
  { CorrectedTranslation }
  { Notes }


\latline
  {dedit. Qui postea, Amulio interfecto, Numitori avo regnum }
  { Translation }
  {150}
  { CorrectedTranslation }
  { Notes }


\latline
  {[\textbf{10}] restituerunt; ipsi pastoribus contractis, civitatem condiderunt, quam }
  { Translation }
  {150}
  { CorrectedTranslation }
  { Notes }


\latline
  {Romulus augurio victor, quod ipse duodecim, Remus sex vultures }
  { Translation }
  {150}
  { CorrectedTranslation }
  { Notes }


\latline
  {viderat, Romam vocavit.  Et, ut eam potius legibus muniret quam }
  { Translation }
  {150}
  { CorrectedTranslation }
  { Notes }


\latline
  {moenibus, edixit, ne quis vallum transiliret.  Quod Remus inridens}
  { Translation }
  {150}
  { CorrectedTranslation }
  { Notes }


\latline
  {transiluit et a Celere centurione rastro fertur occisus esse.}
  { Translation }
  {150}
  { CorrectedTranslation }
  { Notes }





% section chapter_4 (end)