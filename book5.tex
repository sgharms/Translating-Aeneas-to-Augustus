\chapter*{Chapter 5} % (fold)
\label{sec:chapter_5}

\latline
  {Cum Tarquinius Superbus, ultimus rex Romanorum, Ardeam }
  { Since Tarquinius the Proud, the last king of the Romans was besieging }
  {150}
  { CorrectedTranslation }
  { \begin{enumerate}
  	\item obsideo, obsidere, obsidi, obsiditum:  to beseige
  \end{enumerate} }


\latline
  {obsideret, Tarquinius Collatinus, consobrino regis genitus, in contubernio}
  { Ardea, Tarquinius Collatinus, born from the king's cousin, was in the tents}
  {150}
  { CorrectedTranslation }
  { Notes }


\latline
  {filiorum regiorum erat.  Cum forte in liberiore convivio}
  { of the sons of the royal sons.  It came about in the course of a free and easy party }
  {150}
  { CorrectedTranslation }
  { \begin{enumerate}
  	\item convivium, n.:  banquet, feast
  \end{enumerate} }


\latline
  {coniugem suam unus quisque laudaret, placuit eis experiri.  Itaque}
  { that each one was praising his own wife among the party, it pleased them to test this.  Therefore}
  {150}
  { CorrectedTranslation }
  { \begin{enumerate}
  	\item experior, experiri, experitum:  to test, to try
  	\item itaque:  therefore, consequentially
  \end{enumerate} }


\latline
  {[\textbf{5}] equis Romam petunt.  Regias nurus in convivio et luxu deprehendunt.}
  { they went to Rome by horse.  They find all the daughters-in-law of the king at banquet and excess.   }
  {150}
  { CorrectedTranslation }
  { \begin{enumerate}
  	\item nurus, nurus, f:  daughter-in-law
  	\item luxus, us, m:  luxury, soft living
  	\item deprehendo, deprendere, deprendi, deprensus:  catch, seize
  \end{enumerate} }


\latline
  {Inde Collatiam petunt.  Lucretiam inter ancillas in lanificio offendunt.}
  { Then they seek the city of Collatia.  They pester Lucretia among her slaves at wool-working. }
  {150}
  { CorrectedTranslation }
  { \begin{enumerate}
  	\item offendo, offendere, offendi, offendum: bother, vex, annoy
  	\item ancilla, -ae, f:  slave-girl
  \end{enumerate} }


\latline
  {Itaque ea pudicissima iudicatur.  Ad quam corrumpendam Tarquinius}
  { Thereafter she is judged to be the most modest.  In the night Tarquinius Sextus }
  {150}
  { CorrectedTranslation }
  { \begin{enumerate}
  	\item corrumpo, corrumpere, corrupi, corruptus:  spoil, rot
  	\item ad + acc. + participle:  A formula for expressing purpose i.e. ``it ad Lucretiam corrupendam.''  Or, generally, ad + participle to express purpose
  \end{enumerate} }


\latline
  {Sextus noctus Collatiam rediit et iure propinquitatis domum Collatini}
  { returned to Collatia to corrupt her and by the law of kinship came to the }
  {150}
  { CorrectedTranslation }
  { \begin{enumerate}
  	\item noctus : locative for nox, archaic, adverbial
  \end{enumerate} }


\latline
  {venit et, in cubiculum Lucretiae inrumpens, pudicitiam expugnavit.  }
  { house of Collatinus and, rushing into the bedroom of Lucretia, raped the modest woman. }
  {150}
  { CorrectedTranslation }
  { \begin{enumerate}
  	\item expugno (1):  to assault, to attack, seize, rape
  \end{enumerate} }


\latline
  {[\textbf{10}] Illa igitur postero die, advocatis Tricipitino patre et Calatino}
  { Therefore during following day she, with summoned men --- her father Tricipitus and  }
  {150}
  { CorrectedTranslation }
  { Notes }


\latline
  {coniuge, rem exposuit et se cultro, quem veste texerat, occidit.}
  { husband Collatinus --- she exposed the deed and killed herself with a blade which she had hidden in her clothes. }
  {150}
  { CorrectedTranslation }
  { Notes }


\latline
  {Deinde Tricipitinus et Collatinus cum Iunio Bruto, soroe regis}
  { Thereafter Tricipitinus and Collatinus, along with Junius Brutus, son of the  }
  {150}
  { CorrectedTranslation }
  { Notes }


\latline
  {genito, in exitium Tarquiniorum coniuraverunt, eorumque exsilio}
  { king's sister, swore an oath for the destruction of the Tarquinii, and by the exile of them }
  {150}
  { CorrectedTranslation }
  { \begin{enumerate}
  	\item conjuro (1):  To swear an oath, to plot
  \end{enumerate} }


\latline
  {morte Lucretiae vindicaverunt.  Tarquinius Superbus ad Porsennam}
  { avenged the death of Lucretia.  Tarquinius Superbus fled to Porsenna, the king of  }
  {150}
  { CorrectedTranslation }
  { Notes }


\latline
  {[\textbf{15}] Etruriae regem confugit, cuius ope regnum recuperare tentavit.}
  { Eritruria, of whom he sought help to regain his kingdom. }
  {150}
  { CorrectedTranslation }
  { Notes }


\latline
  {Roma pulsus, Cumas concessit, ubi per summam ignominiam}
  { Having been pushed from rome, he yielded to Cumae where he spent the balance }
  {150}
  { CorrectedTranslation }
  { Notes }


\latline
  {reliquum vitae tempus exegit.  Tarquininiis in exsilium actis, Brutus et}
  { of his life in the height of disgrace.  With the Tarquinii having been pushed into exile, Brutus }
  {150}
  { CorrectedTranslation }
  { Notes }


\latline
  {Collatinus primi consules creati sunt.}
  { and Collatinus were created the first consuls. }
  {150}
  { CorrectedTranslation }
  { Notes }





% section chapter_5 (end)
