\chapter*{Chapter 5} % (fold)
\label{sec:chapter_5}

Cum Tarquinius Superbus, ultius rex Romanorum, Ardeam 
obsideret, Tarquinius Callatinus, consobrino regis genitus, in contubernio
filiorum regiorum erat.  Cum forte in liberiore convivio
coniugem suam unus quisque laudaret, placuit eis experiri.  Itaque
equis Romam petunt.  Regias nurus in convivio et luxu deprehendunt.
Inde Collatiam petunt.  Lucretiam inter ancillas in lanificio offendunt.
Itaque ea pudicissima iudicatur.  Ad quam corrumpendam Tarquinius
Sextus noctus Collatiam rediit et iure propinquitatis domum Collatini
venit et, in cubiculum Lucretiae inrumpens, pudicitiam expugnavit.  
Illa igitur postero die, advocatis Tricipitino patre et Calatino
coniuge, rem exposuit et se cultro, quem veste texerat, occidit.
Deinde Tricipitinus et Collatinus cum Iunio Bruto, soroe regis
genito, in exitium Tarquiniorum coniuraverunt, eorumque exsilio
morte Lucretiae vindicaverunt.  Tarquinius Superbus ad Porsennam
Etruriae regem confugit, cuius ope regnum recuperare tentavit.
Roma pulsus, Cumas concessit, ubi per summam ignominiam
reliquum vitae tempus exegit.  Tarquininiis in exsilium actis, Brutus et
Collatinus primi consules creati sunt.



% section chapter_5 (end)
